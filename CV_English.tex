\documentclass[a4paper,12pt,oneside]{letter}

\usepackage{ucs}
\usepackage[utf8x]{inputenc}
\usepackage[T1]{fontenc}
\usepackage{multicol}
\usepackage[english,francais]{babel}
\usepackage{fontenc}
\usepackage[pdftex]{graphicx}

\usepackage[pdftex=true,
hyperindex=true,
colorlinks=true,pdfstartview={Fit}]{hyperref}
\hypersetup{urlcolor=blue}


\usepackage{url}

\setlength{\parindent}{0cm} %Pour le retrait du paragraphe

\usepackage{soul} % Pour souligner ou barrer du texte
\usepackage{ulem}

\usepackage{textcomp}

\usepackage{setspace} % Interligne
\singlespacing
%\onehalfspacing
%\doublespacing

\usepackage{geometry}
\geometry{%
a4paper,
body={180mm,265mm},
left=15mm,top=15mm,
headheight=7mm,headsep=4mm,
marginparsep=4mm,
marginparwidth=11mm}

%\usepackage{eurosym}

%\usepackage{bookman} % Différents packs de police
%\usepackage{charter}
%\usepackage{newcent}
\usepackage{lmodern}
%\usepackage{mathpazo}
%\usepackage{mathptmx}



\usepackage{wrapfig} % pour encadrer les figures avec du texte

\usepackage{multicol} % Pour utiliser l'environnement multicol
\setlength\columnseprule{.4pt} % Pour mettre un trait séparateur de colonnes
\usepackage{multirow} % Pour fusionner les lignes dans les tableaux
\usepackage{array}
\usepackage{rotating} % Pour tourner un tableau
\usepackage{colortbl}
\usepackage{xcolor}



\date{2011-04-12}

\begin{document}

\begin{wrapfigure}[1]{r}{4.5cm}
\includegraphics[width=4.5cm]{ID0.jpg}
\end{wrapfigure}

Name: GAU \hspace{1cm} Surname: Rémi \hfill

Sex: Male

Date of birth: 3\textsuperscript{rd} august 1982

Nationality: French

\begin{tabbing}
\hspace{3cm}\=\kill
Languages:
 \> French (mother tongue)\\ 
 \> English (fluent)\\ 
 \> German (basics)\\ 
%  \> Norwegian (basics)
\end{tabbing}

{Tel (mobile):} (+44) 777 100 9880

{Email:} \href{mailto:remi\textunderscore gau@hotmail.com}{remi\textunderscore gau@hotmail.com}


% \begin{tabular}[!h]
% {p{11cm}||p{7cm}}
% 
% \underline{\textbf{Address of last position:}} 							& \underline{\textbf{Secondary address:}}\\ 
% \href{http://www.birmingham.ac.uk}{University of Birmingham}						& 72 Leahurst Crescent\\
% \href{http://www.birmingham.ac.uk/research/activity/cncr/index.aspx}{\textit{Research Centre for
% Computational Neuroscience and Cognitive Robotics}} 							& Harbourne\\
% Hills Building 												& B17 0LD Birmingham\\
% England 											& England \\
% {Tel:} (+44) 121 414 6032 										& \\
% \end{tabular}

\setlength\minrowclearance{0.2cm}
\setlength\arrayrulewidth{2pt}

{%
\newcolumntype{A}{%
>{\bfseries}%
p{2.6cm}}

\newcolumntype{B}{%
>{\LARGE \centering \columncolor[gray]{.5}[.5\tabcolsep] \bfseries }%
p{17.6cm}}

\begin{tabular}{B}
\underline{CURRENT SITUATION}
\end{tabular}

\begin{tabular}{Ap{14.5cm}}
\textbullet~\underline{Since February 2012:} 	& \hfill \large\textbf{PhD in neurosciences} \\
						& \underline{Laboratories:} \newline
						\href{http://www.kyb.mpg.de}{\textit{Max Planck institute for biological cybernetics}} (Tübingen, Germany) \newline
						\href{http://www.birmingham.ac.uk/research/activity/cncr/index.aspx}{\textit{CNCR}}, University of Birmingham (Birmingham, England) \\
						& \large\underline{Thesis title:} The effect of prior information on multi-sensory integration: a behavioral and neuroimaging study\\
						& \underline{Supervisor:} NOPPENEY Uta \\
						& \underline{Email:} \href{mailto:u.noppeney@bham.ac.uk}{u.noppeney@bham.ac.uk}

\end{tabular}

\begin{tabular}{B}
\underline{ACADEMIC TITLES}
\end{tabular}

% \setlength{\fboxrule}{1pt}
% \framebox[18cm]{\LARGE\textbf{ACADEMIC TITLES}}

\begin{tabular}{Ap{14.5cm}}
\textbullet~\underline{2006-2010:} 	& \hfill \large\textbf{PhD in neurosciences} \hfill Defended in september 2010 \\ 
					& \href{http://www.upmc.fr/}{\textit{University Pierre et Marie Curie}} (Paris, France) \newline
					  Doctoral School « \href{http://ed3c.snv.jussieu.fr/}{Cerveau, Cognition, Comportement} » \newline
					  first class honours \\
					& \large\underline{Thesis title:} Serotonergic neurons of the lateral paragigantocellular nucleus: roles in pain modulation and baroreflex inhibition \\
					& \underline{Laboratory:} \href{http://www.broca.inserm.fr/site_cpn/new/index.php}{Psychiatry and Neurosciences Center} \\
					& \underline{Supervisor:} BERNARD Jean-François
\end{tabular} 

\pagebreak

\begin{center}
\Large\textbf{Graduate studies}
\end{center}

\begin{tabular}{Ap{14.5cm}}
\textbullet~\underline{2005-2006:} & \href{http://www.master.bip.upmc.fr/}{\large\textbf{Master of integrative biology and physiology, option: neurosciences – research specialization}} \newline
				     \normalsize \href{http://www.upmc.fr/}{\textit{University Pierre et Marie Curie}} (Paris, France) \newline
				     second class honours; rank: NA \\
\textbullet~\underline{2004-2005:} & \href{http://www.ups-tlse.fr/PANPS0_71/0/fiche___formation/&RH=rub02}{\large\textbf{Master of neuropsychology – research specialization}} \newline
				     \normalsize \href{http://www.ups-tlse.fr/}{\textit{University Paul Sabatier}} (Toulouse, France)\newline
				     second class honours; rank: NA \\
\textbullet~\underline{2003-2004:} & \large\textbf{Master of cellular biology and animal physiology} \newline
				     \normalsize \href{http://www.mcgill.ca/}{\textit{McGill university}} (Montréal, Canada) through the CREPUQ exchange program with the \href{http://www.univ-montp2.fr/}{\textit{University of Montpellier II}} (France)\newline
				     second class honours; rank: 1/55
\end{tabular} 

\begin{center}
\Large\textbf{Undergraduate studies}
\end{center}

\begin{tabular}{Ap{14.5cm}}
\textbullet~\underline{2002-2003:} & \large\textbf{Licence of cellular biology and animal physiology} \newline
				     \normalsize \href{http://www.univ-montp2.fr/}{\textit{University of Montpellier II}} (France) \newline
				     second class honours; rank: 1/80 \\
\textbullet~\underline{2001-2003:} & \large\textbf{Diplôme d’études universitaires générales of psychology} \newline
				     \normalsize \href{http://www.univ-montp3.fr/}{\textit{University Paul-Valéry}} (Montpellier, France)\newline
				     second class honours; rank: NA \\
\textbullet~\underline{2000-2002:} & \large\textbf{Diplôme d’études universitaires générales of biochemistry and physiology} \newline
				     \normalsize \href{http://www.univ-montp2.fr/}{\textit{University of Montpellier II}} (France)\newline
				     first class honours; rank: 3/250
\end{tabular}


\begin{center}
\Large\textbf{Baccalaureate}
\end{center}

\begin{tabular}{Ap{14.5cm}}
\textbullet~\underline{2000:} & \large\textbf{International Baccalaureate} \newline
				\normalsize \href{http://www.rcnuwc.no/}{\textit{Red Cross Nordic United World College}} (Flekke, Norway) \newline
grade: 36/45 \\
\end{tabular}


\begin{tabular}{B}
\underline{CURRENT WORK}
\end{tabular}

I am currently working on how contextual congruency might affect the multi-sensory integration of audio-visual speech using psychophysics and fMRI. This is part of an ongoing project supported and financed by the Bernstein Center for Computational Neuroscience of Tübingen.


\begin{tabular}{B}
\underline{ACADEMIC APPOINTMENTS}
\end{tabular}

\begin{center}
\Large\textbf{Laboratory trainings}
\end{center}

\begin{minipage}[c]{7.7cm}
\textbullet~\underline{\textbf{2003-2005}}: I conducted the data analysis of a pilot study preliminary to a multicentric rehabilitation program for dyslexic children.
\end{minipage}
\hfill
\begin{minipage}[c]{10cm}
\setlength\minrowclearance{0.2cm}
\setlength\arrayrulewidth{1.5pt}
\small
\begin{tabular}[t]{|l|}\hline
\underline{Laboratory:} Neuroimaging and neurological handicaps\\
INSERM unit 825, Hôpital Purpan, Toulouse, France\\
\underline{Supervisor:} DEMONET Jean-François\\
\underline{Email:} \href{mailto:jean-francois.demonet@inserm.fr}{jean-francois.demonet@inserm.fr}\\
\underline{Tel:} (+33) 5 61 77 95 19 \\ \hline
\end{tabular}
\end{minipage}

\begin{minipage}[c]{6.7cm}
\textbullet~\underline{\textbf{2004:}}	I worked for a month on electrophysiology experiments aimed at better understanding the effects of dopamine on the long term potentiation (LTP) of pyramidal neurons in slices of rat prefrontal cortex.
\end{minipage}
\hspace{3mm}
\begin{minipage}[c]{\textwidth}
\setlength\minrowclearance{0.1cm}
\setlength\arrayrulewidth{1.5pt}
\small
\begin{tabular}[c]{|l|}\hline
\underline{Laboratory:} Laboratoire de biologie des processus adaptatifs\\
\href{http://www.upmc.fr/}{\textit{University Pierre et Marie Curie}}, Paris, France\\
\underline{Supervisor:} OTANI Satoru\\
\underline{Email:} \href{mailto:satoru.otani@snv.jussieu.fr}{satoru.otani@snv.jussieu.fr}\\ \hline
\end{tabular}
\end{minipage}

\begin{minipage}[c]{7.7cm}
\textbullet~\underline{\textbf{2003:}}	I worked for 2 months on a set of experiments testing the effects of prenatal stress or in-utero cocaine injection on learning in young rats and on hippocampic long term potentiation on rat brain slices.
\end{minipage}
\hspace{3mm}
\begin{minipage}[c]{\textwidth}
\setlength\minrowclearance{0.1cm}
\setlength\arrayrulewidth{1.5pt}
\small
\begin{tabular}[c]{|l|}\hline
\underline{Laboratory:} Laboratoire de plasticité cérébrale\\
CNRS-UMR 5102, \href{http://www.univ-montp2.fr/}{\textit{University of Montpellier II}}, France\\
\underline{Supervisor:} VIGNES Michel\\
\underline{Email:} \href{mailto:mvignes@univ-montp2.fr}{mvignes@univ-montp2.fr}\\ \hline
\end{tabular}
\end{minipage}

\begin{tabular}{B}
\underline{TEACHING EXPERIENCES}
\end{tabular}

\textbullet~\underline{\textbf{??? semester 2013-2013-2015:}} 	Teaching assistant Neuroimaging class ?????????

Supervisors: NOPPENEY, U; Contact: ????????????, 

\textbullet~\underline{\textbf{2nd semester 2009:}} 	Teaching assistant for human evolution classes of the preparatory program to paramedical training in the University Pierre et Marie Curie

Supervisors: A. AURENGO \& T. DARRIBERE; Contact: bureau 207 bis, 91 boulevard de l’hôpital, 

75 634 Paris, Tel: (+33) 1 40 77 95 77

\textbullet~\underline{\textbf{1rst semester 2005:}} 	Teaching assistant for psychophysiology classes of the undergraduate psychology program of the University Mirail-Toulouse II

Supervisor: BRETDIBAT Jean-Luc; Email: \href{mailto:bretdiba@univ-tlse2.fr}{bretdiba@univ-tlse2.fr}

\begin{tabular}{B}
\underline{SCHOLARSHIPS}
\end{tabular}

\begin{tabular}{Ap{14.5cm}}
\textbullet~\underline{2006-2009:} & Scholarship of the French ministry of research and technology \\
\textbullet~\underline{2001-2003:} & Scholarship of the French society for the study and treatment of pain \newline
				     (SFETD: \url{http://www.sfetd-douleur.org})
\end{tabular}

\begin{tabular}{B}
\underline{PUBLICATIONS}
\end{tabular}

\begin{center}
\Large\textbf{Peer reviewed papers}
\end{center}

\textbullet~GAU R, SEVOZ-COUCHE C, HAMON M, BERNARD JF; 
\href{http://www.painjournalonline.com/article/S0304-3959\%2812\%2900540-4/abstract}{Noxious stimulation excites serotonergic neurons: a comparison between the lateral paragigantocellular reticular and the raphe magnus nuclei}, \textit{Pain}, 2013 Oct 5, 154 (5): 647-49; doi: 10.1016/j.pain.2012.09.012

\textbullet~GAU R, SEVOZ-COUCHE C, LAGUZZI R, HAMON M, BERNARD JF; \href{http://www.painjournalonline.com/article/S0304-3959\%2809\%2900554-5/abstract}{Inhibition of Baroreflex by Nociception: A Key Role for Lateral Paragigantocellular Serotonergic Cells}, \textit{Pain}, 2009 Dec 5; 146 (3): 315-24; DOI: 10.1016/j.pain.2009.09.018

\textbullet~BERNARD JF, NETZER F, GAU R, HAMON M, LAGUZZI R, SÉVOZ-COUCHE C; \href{http://onlinelibrary.wiley.com/doi/10.1002/cne.21532/abstract;jsessionid=5BCAA74755E291EB3D5A1BF64254181B.d01t02}{Critical role of B3 serotonergic cells in baroreflex inhibition during the defense reaction triggered by dorsal periaqueductal gray stimulation}, \textit{Journal of comparative anatomy}, 2008 Jan 1; 506(1): 108-21; DOI: 10.1002/cne.21532

\begin{center}
\Large\textbf{Posters}
\end{center}

\textbullet~GAU R, TRAMPEL R, BAZIN PL, TURNER R, NOPPENEY U; Effect of sensory modality and attention on layer-specific activations in sensory cortices, Human brain mapping, Hamburg (Germany), 2014, %, 361.10/BB36 

\textbullet~GAU R, NOPPENEY U; The left prefrontal cortex controls information integration by combining bottom-up inputs and top-down predictions, Annual meeting of the Society for neurosciences, San Diego, (California, USA), 2013 %, 361.10/BB36 

\textbullet~BERNARD JF, SEVOZ-COUCHE C, HAMON M, GAU R; Responses of lateral paragigantocellular and raphe magnus serotonergic neurons to noxious stimuli: a comparative reappraisal using juxtacellular recording, 13\textsuperscript{th} world congress on pain; International association for the study of pain; Montréal (Québec, Canada), 2010, PW 205 

\textbullet~BERNARD JF, SEVOZ-COUCHE C, HAMON M, GAU R; Involvement of lateral paragigantocellular reticular serotonergic and non-serotonergic neurons in nociceptive processes, Annual meeting of the Society for neurosciences, Chicago, (Ilinois, USA), 2009, 361.10/BB36

\textbullet~GAU R, SEVOZ-COUCHE C, HAMON M, LAGUZZI R, BERNARD JF; Inhibition of cardiac baroreflex by intense noxious stimuli: a serotonergic mechanism involving the lateral paragigantocellular reticular nuclei; Annual meeting of the Society for neurosciences, Washington (D.C., USA), 2008, 174.22/NN22

\textbullet~BERNARD JF, SEVOZ-COUCHE C, HAMON M, LAGUZZI R, GAU R; Critical role of the B3 group in the baroreflex inhibition evoked by thermal noxious stimulation in the rat, Annual meeting of the Society for neurosciences, San Diego (California, USA), 2007, 724.24/MM1

\textbullet~BERNARD JF, NETZER F, GAU R, HAMON M, LAGUZZI R, SEVOZ-COUCHE C; Serotonergic neurons of B3 group: critical role in baroreflex inhibition during the defense reaction in the rat; Annual meeting of the Society for neurosciences, Atlanta (Georgia, USA), 2006, 356.9/AA9

\begin{center}
\Large\textbf{Talks}
\end{center}

\textbullet~GAU R; \href{http://www.sfetd-douleur.org/ModuleEventPublic/viewPresentation.phtml?about=rc\%2f2010\%2f10econgres\%2f10e-congre\%2fsession\%2f20101118-0845-20\%2f1015-62\%2f_container}{Implication du groupe sérotoninergique B3 dans le contrôle des circuits de la douleur et des réactions neurovégétatives associées}; 10\textsuperscript{th} congress of the French society for the study and treatment of pain (SFETD: \url{http://www.sfetd-douleur.org}), Marseille (France), 18\textsuperscript{th} november 2010

\begin{tabular}{B}
\underline{SKILLS}
\end{tabular}

\begin{description}
\item[\textbullet~Human neuroimaging (using SPM):] PET scan \& MRI images preprocessing and analyses (individual, group with random and fixed effects models, correlation analysis to in-scan or out of scan behavioral data) for both block and event-related designs.\\

\item [\textbullet~Softwares:] Proficient use \& knowledge of: matlab, PsychToolBox, Presentation, SPM, UNIX operating system, Freesurfer, Spike 2 (electrophysiology), MIPAV, Freehand \& Inkscape (vectorial drawing), Photoshop \& Gimp, Latex, Microsoft/Libre Office. Good use \& knowledge of different statistic analysis softwares (SPSS, statisticA, Statview, SigmaPlot), Image J. Some use \& knowledge MRIcro, C/C++.

\item [\textbullet~Electrophysiology:] \textit{In vivo} extracellular recording combined coupled to juxtacellular labeling in halothane anesthetized rats, LTP protocols and associated pharmacological modulation on rat brain slices using intracellular recording in current clamp or extracellular recording with a multi-electrodes array. 

\item [\textbullet~Functional neuroanatomy:] Use of neuroanatomical tracers (Phaseolus, TMR, fluorogold), functional c-fos expression experiments, stereotaxic local pharmacological neuroinactivation combined with direct online physiological parameters (heart rate, blood pressure...) recording and analysis (using spike 2) in anesthetized rats. 

\item [\textbullet~Microscopy \& histology:] Total animal fixation with formalin \& brain extraction, general histological techniques (Cresyl violet, thionin), double immunohistochemical and immunofluorescent labeling, epifluorescent transmission and confocal microscope images acquisition and processing, some notions of in situ hybridization.

\end{description}
% 
% \begin{tabular}{B}
% \underline{OTHER INTERESTS}
% \end{tabular}
% 
% \begin{itemize}
% \item Skepticism, history \& philosophy of sciences, evolution
% \item Gender studies, sociology and political economy
% \item Rock climbing (bouldering and traditional), running, telemark skiing 
% \item Literature (science fiction) 
% \item Music (classical, jazz, blues, metal)
% \end{itemize}
% 
}%

\end{document}
