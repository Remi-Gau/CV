\documentclass[a4paper,12pt,oneside]{letter}

\usepackage{ucs}
\usepackage[utf8x]{inputenc}
\usepackage[T1]{fontenc}
\usepackage{multicol}
\usepackage[english]{babel}
\usepackage{fontenc}
\usepackage[pdftex]{graphicx}

\usepackage[pdftex=true,
hyperindex=true,
colorlinks=true,pdfstartview={Fit}]{hyperref}
\hypersetup{urlcolor=blue}


\usepackage{url}

\setlength{\parindent}{0cm} %Pour le retrait du paragraphe

\usepackage{soul} % Pour souligner ou barrer du texte
\usepackage{ulem}

\usepackage{textcomp}

\usepackage{setspace} % Interligne
\singlespacing
%\onehalfspacing
%\doublespacing

\usepackage{geometry}
\geometry{%
a4paper,
body={180mm,265mm},
left=15mm,top=15mm,
headheight=7mm,headsep=4mm,
marginparsep=4mm,
marginparwidth=11mm}

%\usepackage{eurosym}

%\usepackage{bookman} % Différents packs de police
%\usepackage{charter}
%\usepackage{newcent}
\usepackage{lmodern}
%\usepackage{mathpazo}
%\usepackage{mathptmx}



\usepackage{wrapfig} % pour encadrer les figures avec du texte

\usepackage{multicol} % Pour utiliser l'environnement multicol
\setlength\columnseprule{.4pt} % Pour mettre un trait séparateur de colonnes
\usepackage{multirow} % Pour fusionner les lignes dans les tableaux
\usepackage{array}
\usepackage{rotating} % Pour tourner un tableau
\usepackage{colortbl}
\usepackage{xcolor}



\date{2017-01-11}

\begin{document}

\newcommand{\myheader}{
\begin{tabular*}{7in}{l@{\extracolsep{\fill}}r}

	\textbf{{\LARGE Rémi Gau}}  & \\
	\hline\hline	\\ %Date of birth: 3\textsuperscript{rd} August 1982 & Nationality: French\\
	%\\
	Languages: French, English, German (basics)\\
	\\
	71 Milner road, B29 7RL & +44 (0) 777 100 9880 \\
	Birmingham, United Kingdom & \href{mailto:remi\textunderscore gau@hotmail.com}{remi\textunderscore gau@hotmail.com} \\
	\\
	\href{http://www.researchgate.net/profile/Remi\textunderscore Gau}{researchgate.net/profile/Remi\textunderscore Gau} & \href{http://www.twitter.com/RemiGau}{@RemiGau} \\
	\\
	\hline\hline
	\end{tabular*}}

\myheader

\setlength\minrowclearance{0.2cm}
\setlength\arrayrulewidth{2pt}

{%

\newcolumntype{A}{%
>{\bfseries}%
p{2.6cm}}

\newcolumntype{B}{%
>{\Large \centering \columncolor[gray]{.9}[.5\tabcolsep] \bfseries }%
p{17.6cm}}



\medskip 

\begin{tabular}{B}
\underline{CURRENT POSITION}
\end{tabular}

\begin{tabular}{Ap{14.5cm}}
\textbullet~\underline{Since February 2012:} 	& \hfill \large\textbf{PhD in Neuroscience} \\
						& \underline{Laboratories:} \newline
						\textit{Computational Cognitive Neuroimaging Laboratory}, School of psychology, University of Birmingham, UK \newline
						registered at the \textit{International Max Planck Research School}, Graduate training centre of neuroscience, Tübingen, Germany \\
						& \underline{Thesis title:} Cortical laminar analyis of multisensory integration using high-field fMRI\\
						& \underline{Supervisor:} Pr. Dr. MD. Uta Noppeney \underline{Email:} \href{mailto:u.noppeney@bham.ac.uk}{u.noppeney@bham.ac.uk}
\end{tabular}


\medskip 

This second PhD focused on the study of the neural correlates of multisensory integration using psychophysics and functional MRI in humans. 

I conducted studies to investigate:
\begin{itemize}
 \item the neural correlates of processing contextual sensory congruency and its effects on new incoming audio-visual speech inputs,
 \item how attention modulates the integration of auditory and visual stimuli across the depth of the cortex in primary and secondary sensory cortices,
 \item at which cortical depth a primary sensory region is best able to distinguish between two non-preferred sensory inputs.
\end{itemize}

%\medskip 
\pagebreak

\begin{tabular}{B}
\underline{ACADEMIC QUALIFICATIONS}
\end{tabular}


\begin{tabular}{Ap{14.5cm}}
\textbullet~\underline{2006-2010:} 	& \hfill \large\textbf{PhD in neuroscience} \hfill Defended in September 2010 \\ 
					& \textit{University Pierre et Marie Curie}, Paris, France \newline
					  Doctoral School « Cerveau, Cognition, Comportement » \newline
					  first class honours \\
					& \underline{Thesis title:} Serotonergic neurons of the lateral paragigantocellular nucleus: roles in pain modulation and baroreflex inhibition \\
					& \underline{Laboratory:} Psychiatry and Neurosciences Center \\
					& \underline{Supervisor:} Dr. MD. Jean-François Bernard 
\end{tabular} 


\medskip 

The first Phd focused on the study of pain, its modulation and its associated autonomous responses in rats. 

My work included:
\begin{itemize}
\item the \textit{in-vivo} single unit electrophysiological recording of serotonergic neurons of the rostroventromedial medulla (RVM) and their response to noxious stimulation. 
\item the  stereotatically directed pharmacological inactivation of these neurons combined with direct online cardiovascular parameters (heart rate, blood pressure) recording, as well as immunohistochemical techniques (neuroanatomical tracers, functional c-fos expression) to assess the role of the RVM in cardiovascular regulations. 
\end{itemize}

%These different lines of work allowed us to conclude that serotonergic neurons of the RVM are more heterogeneous and more responsive to noxious stimulation than was previously thought and that this argues in favor of their role, not only in pain modulation, but also in the autonomic responses usually associated with noxious stimuli.

%\pagebreak

\begin{center}
\large\textbf{Graduate qualifications}
\end{center}

\begin{tabular}{Ap{14.5cm}}
\textbullet~\underline{2005-2006:} & \large\textbf{Master of Integrative Biology and Physiology: Neurosciences (research specialization)} \newline
				     \normalsize \textit{University Pierre et Marie Curie}, Paris, France \newline
				     Second class honours ; rank: NA\\ %; rank: NA
\textbullet~\underline{2004-2005:} & \large\textbf{Master of Neuropsychology (research specialization)} \newline
				     \normalsize \textit{University Paul Sabatier}, Toulouse, France\newline
				     Second class honours ; rank: NA \\ %; rank: NA
\textbullet~\underline{2003-2004:} & \large\textbf{Master of Cellular Biology and Animal Physiology} \newline
				     \normalsize \textit{McGill university},Montréal, Canada, through the CREPUQ exchange program with the University of Montpellier II, France\newline
				     Second class honours; rank: 1/55
\end{tabular} 


\pagebreak

\begin{center}
\large\textbf{Undergraduate qualifications}
\end{center}

\begin{tabular}{Ap{14.5cm}}
\textbullet~\underline{2002-2003:} & \large\textbf{License of Cellular Biology and Animal Physiology} \newline
				     \normalsize \textit{University of Montpellier II}, France \newline
				     Second class honours; rank: 1/80 \\
\textbullet~\underline{2001-2003:} & \large\textbf{Diplôme d’Etudes Universitaires Générales of Psychology} \newline
				     \normalsize \textit{University Paul-Valéry}, Montpellier, France\newline
				     Second class honours ; rank: NA \\ %; rank: NA
\textbullet~\underline{2000-2002:} & \large\textbf{Diplôme d’Etudes Universitaires Générales of Biochemistry and Physiology} \newline
				     \normalsize \textit{University of Montpellier II}, France\newline
				     First class honours; rank: 3/250
\end{tabular}


\begin{center}
\large\textbf{Baccalaureate}
\end{center}

\begin{tabular}{Ap{14.5cm}}
\textbullet~\underline{2000:} & \large\textbf{International Baccalaureate} \newline
				\normalsize Red Cross Nordic United World College, Flekke, Norway \newline
				Grade: 36/45 \\
\end{tabular}


\medskip 

\begin{tabular}{B}
\underline{PUBLICATIONS}
\end{tabular}


\begin{center}
\large\textbf{Peer reviewed papers}
\end{center}

\textbullet~\textbf{Gau R}, Noppeney, U; 
\href{http://www.sciencedirect.com/science/article/pii/S1053811915008605}{How prior expectations shape multisensory perception}, \textit{Neuroimage}, 2016; DOI: 10.1016/j.neuroimage.2015.09.045

\textbullet~\textbf{Gau R}, Sevoz-Couche C, Hamon M, Bernard JF; 
\href{http://www.researchgate.net/profile/Remi_Gau/publication/233394350_Noxious_stimulation_excites_serotonergic_neurons_A_comparison_between_the_lateral_paragigantocellular_reticular_and_the_raphe_magnus_nuclei/links/586bf62508aebf17d3a5b232.pdf}{Noxious stimulation excites serotonergic neurons: a comparison between the lateral paragigantocellular reticular and the raphe magnus nuclei}, \textit{Pain}, 2013; DOI: 10.1016/j.pain.2012.09.012

\textbullet~\textbf{Gau R}, Sevoz-Couche C, Laguzzi R, Hamon M, Bernard JF; \href{http://www.researchgate.net/profile/Remi_Gau/publication/38057329_Inhibition_of_cardiac_baroreflex_by_noxious_thermal_stimuli_A_key_role_for_lateral_paragigantocellular_serotonergic_cells/links/586bf90d08ae8fce4919e188.pdf}{Inhibition of Baroreflex by Nociception: A Key Role for Lateral Paragigantocellular Serotonergic Cells}, \textit{Pain}, 2009; \\DOI: 10.1016/j.pain.2009.09.018

\textbullet~Bernard JF, Netzer F, \textbf{Gau R}, Hamon M, Laguzzi R, Sevoz-Couche C; 
\href{http://www.researchgate.net/profile/Remi_Gau/publication/5856955_Critical_role_of_B3_serotonergic_cells_in_baroreflex_inhibition_during_the_defense_reaction_triggered_by_dorsal_periaqueductal_gray_stimulation/links/586bf81d08ae6eb871bb6f47.pdf}{Critical role of B3 serotonergic cells in baroreflex inhibition during the defense reaction triggered by dorsal periaqueductal gray stimulation}, \textit{Journal of comparative anatomy}, 2008; DOI: 10.1002/cne.21532


\begin{center}
\large\textbf{Posters}
\end{center}

\textbullet~\textbf{Gau R}, Trampel R, Bazin PL, Turner R, Noppeney U; \href{http://www.researchgate.net/profile/Remi_Gau/publication/312040907_Layer-specific_attentional_modulation_and_multisensory_interactions_in_sensory_cortices/links/586bec8108ae8fce4919e07e.pdf?origin=publication_detail&ev=pub_int_prw_xdl&msrp=ebfE-ttTbrYfVzAofje5aX7FDjiW85HC3yqk8seJjhsOoLqmeRviQBtfDuXmv_czDArKxDp2vM32swxlbQzyvbVTbKL6xRpsEZ-BeqgslOo.KWD8vv500DrprbLnXSrDxinWDQrliScdbic1rjyyqLUG12KOwbjri9jbFDwSYWMkGLoSocKyAei7eQLoDCU9Dw.4eQVGmpcVG-AAQDI4KiYt-xA9SDkF_u7xvWOznoi52trnfagq5aJVIAgpfhgitdzt3LYuUPdJ8gE6jmG5I94Nw}{Layer-specific attentional modulation and multisensory interactions in sensory cortices}, Annual Meeting of the Organization for Human Brain Mapping, Geneva (Switzerland), 2016; DOI: 10.13140/RG.2.2.32219.57124

\textbullet~\textbf{Gau R}, Trampel R, Bazin PL, Turner R, Noppeney U; \href{http://www.researchgate.net/profile/Remi_Gau/publication/312040995_Effect_of_sensory_modality_and_attention_on_layer-specific_activations_in_sensory_cortices/links/586bef4108ae329d621216ff.pdf?origin=publication_detail&ev=pub_int_prw_xdl&msrp=rWGQkC7vAu_P6H2YxCI3M31egmAPzpbXdoRN1OdTlFE2ytvRyytc2DItV2mPnvlGPXAOT-Q0hOx0cWKEfhvF_ZK7oOzQtXTOX1cxj1f7wdc.EjFU3lhxIUR_qptyNrCYqZDXVg6nG7zWe8VL0WxLSdSMUGnjz_v4L5Yov-RWctdEwDPFRXLasy1FBdCk7Dlydw.IANTdlTw5MPB88HQJNdTi7lFv6Sygddbknct3mMTSsgH20heYrYcKnbsQLxtcsNd-HHWZUEIEP4-7bkQLNDUhQ}{Effect of sensory modality and attention on layer-specific activations in sensory cortices}, Annual Meeting of the Organization for Human Brain Mapping, Hamburg (Germany), 2014; DOI: 10.13140/RG.2.2.18797.79846

\textbullet~\textbf{Gau R}, Noppeney U; \href{http://www.researchgate.net/profile/Remi_Gau/publication/312040995_Effect_of_sensory_modality_and_attention_on_layer-specific_activations_in_sensory_cortices/links/586bef4108ae329d621216ff.pdf?origin=publication_detail&ev=pub_int_prw_xdl&msrp=rWGQkC7vAu_P6H2YxCI3M31egmAPzpbXdoRN1OdTlFE2ytvRyytc2DItV2mPnvlGPXAOT-Q0hOx0cWKEfhvF_ZK7oOzQtXTOX1cxj1f7wdc.EjFU3lhxIUR_qptyNrCYqZDXVg6nG7zWe8VL0WxLSdSMUGnjz_v4L5Yov-RWctdEwDPFRXLasy1FBdCk7Dlydw.IANTdlTw5MPB88HQJNdTi7lFv6Sygddbknct3mMTSsgH20heYrYcKnbsQLxtcsNd-HHWZUEIEP4-7bkQLNDUhQ}{The left prefrontal cortex controls information integration by combining bottom-up inputs and top-down predictions}, Annual meeting of the Society for neurosciences, San Diego, (California, USA), 2013; DOI: 10.13140/RG.2.2.12086.91207

\textbullet~Bernard JF, Sevoz-Couche C, Hamon M, \textbf{Gau R}; Responses of lateral paragigantocellular and raphe magnus serotonergic neurons to noxious stimuli: a comparative reappraisal using juxtacellular recording, 13\textsuperscript{th} world congress on pain; International association for the study of pain; Montréal (Québec, Canada), 2010

\textbullet~Bernard JF, Sevoz-Couche C, Hamon M, \textbf{Gau R}; Involvement of lateral paragigantocellular reticular serotonergic and non-serotonergic neurons in nociceptive processes, Annual meeting of the Society for neurosciences, Chicago, (Ilinois, USA), 2009

\textbullet~\textbf{Gau R}, Sevoz-Couche C, Hamon M, Laguzzi R, Bernard JF; Inhibition of cardiac baroreflex by intense noxious stimuli: a serotonergic mechanism involving the lateral paragigantocellular reticular nuclei; Annual meeting of the Society for neurosciences, Washington (D.C., USA), 2008

\textbullet~Bernard JF, Sevoz-Couche C, Hamon M, Laguzzi R, \textbf{Gau R}; Critical role of the B3 group in the baroreflex inhibition evoked by thermal noxious stimulation in the rat, Annual meeting of the Society for neurosciences, San Diego (California, USA), 2007

\textbullet~Bernard JF, Netzer F, \textbf{Gau R}, Hamon M, Laguzzi R, Sevoz-Couche C; Serotonergic neurons of B3 group: critical role in baroreflex inhibition during the defense reaction in the rat; Annual meeting of the Society for neurosciences, Atlanta (Georgia, USA), 2006


\begin{center}
\large\textbf{Talks}
\end{center}

\textbullet~\textbf{Gau R}; \href{http://www.researchgate.net/profile/Remi_Gau/publication/312084405_Implication_du_groupe_serotoninergique_B3_dans_le_controle_des_circuits_de_la_douleur_et_des_reactions_neurovegetatives_associees/links/586e5f4a08ae6eb871bcfcc2?origin=publication_detail&ev=pub_int_prw_xdl&msrp=fq7qspniFbcYB53CbrroAtz2DNXxUJar7CmckJaw6O2YzI7LWasFAW99t9G74Z5sL-KLoJT2Dqv8D075R3ghy55Fy6xPne3_ARZvM87O3Ao.5ii75N_9y4vKCns-VUuTTDNhKb6ruhYALqaObgtdOB4WIkUSArh9tdzN9AtPRWz68YGcjTb3scxfWZy6tYxnSg.naSDUCbUZHM6gF6BaFbjpF3ARcyIPTBh1UsuA1cVoOZQsoPMn9YGNC2UsgdGA0P28xoO8bXY1221-tRVtfrUZA}{Implication du groupe sérotoninergique B3 dans le contrôle des circuits de la douleur et des réactions neurovégétatives associées}; 10\textsuperscript{th} congress of the French society for the study and treatment of pain, Marseille (France), 18\textsuperscript{th} November 2010


\medskip 

\begin{tabular}{B}
\underline{TEACHING EXPERIENCE}
\end{tabular}

\textbullet~\underline{\textbf{2\textsuperscript{nd} Semester of 2013 to 2016:}} 	Teaching assistant for the \textit{Advanced Brain Imaging} masters course at the University of Birmingham (UK)

Supervisor: Noppeney, U; Email: \href{mailto:u.noppeney@bham.ac.uk}{u.noppeney@bham.ac.uk}

\textbullet~\underline{\textbf{2\textsuperscript{nd} Semester 2009:}} 	Teaching assistant for the \textit{Human Evolution} class of the preparatory program to paramedical training at the University Pierre et Marie Curie (Paris, France)

Supervisors: Aurengo, A \& Darribere, T; Tel: (+33) 1 40 77 95 77

\textbullet~\underline{\textbf{1\textsuperscript{rst} Semester 2005:}} 	Teaching assistant for the \textit{Psychophysiology} undergraduate class of the psychology program of the University Mirail-Toulouse II (France)

Supervisor: Bretdibat JL; Email: \href{mailto:bretdiba@univ-tlse2.fr}{bretdiba@univ-tlse2.fr}


\medskip 

\begin{tabular}{B}
\underline{RESEARCH AND TECHNICAL SKILLS}
\end{tabular}


\textbullet~\textbf{Human neuroimaging:} design, implementation and analysis of structural and functional MRI studies, fMRI design efficiency optimization, preprocessing, mass univariate and multivariate pattern analysis, high-field structural and functional MRI analyses

\textbullet~\textbf{Cognitive neuroscience:} design, implementation and analysis of psychophysics and eyetracking studies of multi-sensory integration 


\pagebreak

\textbullet~\textbf{Information technology:}

\begin{center}
\begin{tabular}[c]{|c|c|c|}
\hline\hline
Program & Area & Knowledge \\
\hline
Matlab & Programming & Excellent \\ 
SPM & Neuroimaging analysis & Excellent \\
Freesurfer & Neuroimaging analysis & Excellent \\
CBS tools & High-field neuroimaging analysis & Excellent \\
LIBSVM & Multivariate pattern analysis & Excellent \\
PsychToolBox & Stimulus presentation & Excellent \\
Presentation & Stimulus presentation & Excellent \\
Bash scripting & Programming & Good \\
cvMANOVA & Multivariate pattern analysis & Good \\
Palamedes & Psychophysics data analysis & Good \\
Spike 2 & Physiological data recordings & Good \\
SPSS & Statistical analysis & Good \\
\LaTeX & Document production & Good \\
R & Programming & Basic \\
C/C++ & Programming & Basic \\
FSL & Neuroimaging analysis & Basic \\
Pronto & Multivariate pattern analysis & Basic \\
TDT & Multivariate pattern analysis & Basic \\
Git/GitHub & Version control & Basic \\
\hline\hline
\end{tabular} 
\end{center}

\textbullet~\textbf{Functional neuroanatomy:} direct online physiological parameters recording and analysis, use of neuroanatomical tracers (Phaseolus, TMR, fluorogold), functional c-fos expression experiments, stereotaxic local pharmacological neuroinactivation. 

\textbullet~\textbf{Electrophysiology:} \textit{In vivo} extracellular recording combined to juxtacellular labeling in halothane anesthetized rats, LTP protocols and associated pharmacological modulation on rat brain slices using intracellular recording in current clamp or extracellular recording with a multi-electrodes array. 

\textbullet~\textbf{Microscopy \& histology:} Total animal fixation with formalin \& brain extraction, general histological techniques (Cresyl violet, thionin), double immunohistochemical and immunofluorescent labeling, epifluorescent transmission and confocal microscope images acquisition and processing.


\medskip 

\begin{tabular}{B}
\underline{PROFESSIONAL ACTIVITIES}
\end{tabular}


\href{http://publons.com/author/1205193/remi-gau#profile}{\textbf{Journal referee for} }

\textbullet~Journal of Experimental Neuroscience

\textbullet~Proceedings of the Annual Conference of the Cognitive Science Society


\textbf{Societies membership}

\textbullet~Society for Neuroscience

\textbullet~Organization for Human Brain Mapping


\textbf{Outreach}

\textbullet~ \href{https://dx.doi.org/10.6084/m9.figshare.4535423.v1}{Our brain, our senses, and us}; Interactive presentation given at the children section of the Skeptics in the Pub of Gravesend (UK); 20\textsuperscript{th} October 2015


\textbf{Internal discussion}

\textbullet~ \href{https://dx.doi.org/10.6084/m9.figshare.4257992.v1}{Brain droppings on the replication crisis in psychology}; talk given at the post-graduate seminar of the Shcool of Psychology at the University of Birmingham (UK); 24\textsuperscript{th} November 2016


\textbf{Others}

\textbullet~ Co-organizer of the EiF Journal Club on methodology and replicability in psychology at the University of Birmingham (UK)



\medskip 

\begin{tabular}{B}
\underline{SCHOLARSHIPS \& AWARDS}
\end{tabular}

\begin{tabular}{Ap{14.5cm}}
\textbullet~\underline{2009-2010:} & Award from the French society for the study and treatment of pain\\
     
\textbullet~\underline{2006-2009:} & Scholarship from the French ministry of research and technology \\
\end{tabular}



\medskip 

\begin{tabular}{B}
\underline{OTHER ACADEMIC APPOINTMENTS}
\end{tabular}

\begin{center}
\large\textbf{Laboratory training}
\end{center}

\begin{minipage}[c]{7.7cm}
\textbullet~\underline{\textbf{2003-2005}}: Conducted the data analysis of a pilot study preliminary to a multicentric rehabilitation program for dyslexic children.
\end{minipage}
\hfill
\begin{minipage}[c]{10cm}
\setlength\minrowclearance{0.2cm}
\setlength\arrayrulewidth{1.5pt}
\small
\begin{tabular}[t]{|l|}\hline
\underline{Laboratory:} Neuroimaging and neurological handicaps\\
INSERM unit 825, Hôpital Purpan, Toulouse, France\\
\underline{Supervisor:} Dr. MD. Demonet JF\\
\underline{Email:} \href{mailto:jean-francois.demonet@inserm.fr}{jean-francois.demonet@inserm.fr}\\
\hline
\end{tabular}
\end{minipage}

\begin{minipage}[c]{6.7cm}
\textbullet~\underline{\textbf{2004:}}	Worked on electrophysiology experiments aimed at better understanding the effects of dopamine on the long term potentiation (LTP) of pyramidal neurons in slices of rat prefrontal cortex.
\end{minipage}
\hspace{3mm}
\begin{minipage}[c]{\textwidth}
\setlength\minrowclearance{0.1cm}
\setlength\arrayrulewidth{1.5pt}
\small
\begin{tabular}[c]{|l|}\hline
\underline{Laboratory:} Laboratoire de biologie des processus adaptatifs\\
University Pierre et Marie Curie, Paris, France\\
\underline{Supervisor:} Dr. Otani S\\
\underline{Email:} \href{mailto:satoru.otani@snv.jussieu.fr}{satoru.otani@snv.jussieu.fr}\\ \hline
\end{tabular}
\end{minipage}

\begin{minipage}[c]{7.7cm}
\textbullet~\underline{\textbf{2003:}}	Designed and applied a series of experiments examining the effects of prenatal stress or in-utero cocaine injection on learning in young rats and on hippocampic long term potentiation on rat brain slices.
\end{minipage}
\hspace{3mm}
\begin{minipage}[c]{\textwidth}
\setlength\minrowclearance{0.1cm}
\setlength\arrayrulewidth{1.5pt}
\small
\begin{tabular}[c]{|l|}\hline
\underline{Laboratory:} Laboratoire de plasticité cérébrale\\
CNRS-UMR 5102, University of Montpellier II, France\\
\underline{Supervisor:} Dr. Vignes M\\
\underline{Email:} \href{mailto:mvignes@univ-montp2.fr}{mvignes@univ-montp2.fr}\\ \hline
\end{tabular}
\end{minipage}
 

}

\end{document}
\grid
